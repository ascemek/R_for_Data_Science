% Options for packages loaded elsewhere
\PassOptionsToPackage{unicode}{hyperref}
\PassOptionsToPackage{hyphens}{url}
%
\documentclass[
]{article}
\usepackage{amsmath,amssymb}
\usepackage{lmodern}
\usepackage{ifxetex,ifluatex}
\ifnum 0\ifxetex 1\fi\ifluatex 1\fi=0 % if pdftex
  \usepackage[T1]{fontenc}
  \usepackage[utf8]{inputenc}
  \usepackage{textcomp} % provide euro and other symbols
\else % if luatex or xetex
  \usepackage{unicode-math}
  \defaultfontfeatures{Scale=MatchLowercase}
  \defaultfontfeatures[\rmfamily]{Ligatures=TeX,Scale=1}
\fi
% Use upquote if available, for straight quotes in verbatim environments
\IfFileExists{upquote.sty}{\usepackage{upquote}}{}
\IfFileExists{microtype.sty}{% use microtype if available
  \usepackage[]{microtype}
  \UseMicrotypeSet[protrusion]{basicmath} % disable protrusion for tt fonts
}{}
\makeatletter
\@ifundefined{KOMAClassName}{% if non-KOMA class
  \IfFileExists{parskip.sty}{%
    \usepackage{parskip}
  }{% else
    \setlength{\parindent}{0pt}
    \setlength{\parskip}{6pt plus 2pt minus 1pt}}
}{% if KOMA class
  \KOMAoptions{parskip=half}}
\makeatother
\usepackage{xcolor}
\IfFileExists{xurl.sty}{\usepackage{xurl}}{} % add URL line breaks if available
\IfFileExists{bookmark.sty}{\usepackage{bookmark}}{\usepackage{hyperref}}
\hypersetup{
  pdftitle={SPR Week 1 Exercises},
  pdfauthor={Sami Cemek},
  hidelinks,
  pdfcreator={LaTeX via pandoc}}
\urlstyle{same} % disable monospaced font for URLs
\usepackage[margin=1in]{geometry}
\usepackage{color}
\usepackage{fancyvrb}
\newcommand{\VerbBar}{|}
\newcommand{\VERB}{\Verb[commandchars=\\\{\}]}
\DefineVerbatimEnvironment{Highlighting}{Verbatim}{commandchars=\\\{\}}
% Add ',fontsize=\small' for more characters per line
\usepackage{framed}
\definecolor{shadecolor}{RGB}{248,248,248}
\newenvironment{Shaded}{\begin{snugshade}}{\end{snugshade}}
\newcommand{\AlertTok}[1]{\textcolor[rgb]{0.94,0.16,0.16}{#1}}
\newcommand{\AnnotationTok}[1]{\textcolor[rgb]{0.56,0.35,0.01}{\textbf{\textit{#1}}}}
\newcommand{\AttributeTok}[1]{\textcolor[rgb]{0.77,0.63,0.00}{#1}}
\newcommand{\BaseNTok}[1]{\textcolor[rgb]{0.00,0.00,0.81}{#1}}
\newcommand{\BuiltInTok}[1]{#1}
\newcommand{\CharTok}[1]{\textcolor[rgb]{0.31,0.60,0.02}{#1}}
\newcommand{\CommentTok}[1]{\textcolor[rgb]{0.56,0.35,0.01}{\textit{#1}}}
\newcommand{\CommentVarTok}[1]{\textcolor[rgb]{0.56,0.35,0.01}{\textbf{\textit{#1}}}}
\newcommand{\ConstantTok}[1]{\textcolor[rgb]{0.00,0.00,0.00}{#1}}
\newcommand{\ControlFlowTok}[1]{\textcolor[rgb]{0.13,0.29,0.53}{\textbf{#1}}}
\newcommand{\DataTypeTok}[1]{\textcolor[rgb]{0.13,0.29,0.53}{#1}}
\newcommand{\DecValTok}[1]{\textcolor[rgb]{0.00,0.00,0.81}{#1}}
\newcommand{\DocumentationTok}[1]{\textcolor[rgb]{0.56,0.35,0.01}{\textbf{\textit{#1}}}}
\newcommand{\ErrorTok}[1]{\textcolor[rgb]{0.64,0.00,0.00}{\textbf{#1}}}
\newcommand{\ExtensionTok}[1]{#1}
\newcommand{\FloatTok}[1]{\textcolor[rgb]{0.00,0.00,0.81}{#1}}
\newcommand{\FunctionTok}[1]{\textcolor[rgb]{0.00,0.00,0.00}{#1}}
\newcommand{\ImportTok}[1]{#1}
\newcommand{\InformationTok}[1]{\textcolor[rgb]{0.56,0.35,0.01}{\textbf{\textit{#1}}}}
\newcommand{\KeywordTok}[1]{\textcolor[rgb]{0.13,0.29,0.53}{\textbf{#1}}}
\newcommand{\NormalTok}[1]{#1}
\newcommand{\OperatorTok}[1]{\textcolor[rgb]{0.81,0.36,0.00}{\textbf{#1}}}
\newcommand{\OtherTok}[1]{\textcolor[rgb]{0.56,0.35,0.01}{#1}}
\newcommand{\PreprocessorTok}[1]{\textcolor[rgb]{0.56,0.35,0.01}{\textit{#1}}}
\newcommand{\RegionMarkerTok}[1]{#1}
\newcommand{\SpecialCharTok}[1]{\textcolor[rgb]{0.00,0.00,0.00}{#1}}
\newcommand{\SpecialStringTok}[1]{\textcolor[rgb]{0.31,0.60,0.02}{#1}}
\newcommand{\StringTok}[1]{\textcolor[rgb]{0.31,0.60,0.02}{#1}}
\newcommand{\VariableTok}[1]{\textcolor[rgb]{0.00,0.00,0.00}{#1}}
\newcommand{\VerbatimStringTok}[1]{\textcolor[rgb]{0.31,0.60,0.02}{#1}}
\newcommand{\WarningTok}[1]{\textcolor[rgb]{0.56,0.35,0.01}{\textbf{\textit{#1}}}}
\usepackage{longtable,booktabs,array}
\usepackage{calc} % for calculating minipage widths
% Correct order of tables after \paragraph or \subparagraph
\usepackage{etoolbox}
\makeatletter
\patchcmd\longtable{\par}{\if@noskipsec\mbox{}\fi\par}{}{}
\makeatother
% Allow footnotes in longtable head/foot
\IfFileExists{footnotehyper.sty}{\usepackage{footnotehyper}}{\usepackage{footnote}}
\makesavenoteenv{longtable}
\usepackage{graphicx}
\makeatletter
\def\maxwidth{\ifdim\Gin@nat@width>\linewidth\linewidth\else\Gin@nat@width\fi}
\def\maxheight{\ifdim\Gin@nat@height>\textheight\textheight\else\Gin@nat@height\fi}
\makeatother
% Scale images if necessary, so that they will not overflow the page
% margins by default, and it is still possible to overwrite the defaults
% using explicit options in \includegraphics[width, height, ...]{}
\setkeys{Gin}{width=\maxwidth,height=\maxheight,keepaspectratio}
% Set default figure placement to htbp
\makeatletter
\def\fps@figure{htbp}
\makeatother
\setlength{\emergencystretch}{3em} % prevent overfull lines
\providecommand{\tightlist}{%
  \setlength{\itemsep}{0pt}\setlength{\parskip}{0pt}}
\setcounter{secnumdepth}{-\maxdimen} % remove section numbering
\ifluatex
  \usepackage{selnolig}  % disable illegal ligatures
\fi

\title{SPR Week 1 Exercises}
\author{Sami Cemek}
\date{}

\begin{document}
\maketitle

\hypertarget{vector-subsetting}{%
\section{Vector subsetting}\label{vector-subsetting}}

A local animal rescue group is trying to track the effectiveness of
their social media presence; they are currently interested in tracking
follower growth. The table below summarizes the number of page likes or
new followers each day:

\begin{longtable}[]{@{}llllllll@{}}
\toprule
& Sun & Mon & Tues & Wed & Thurs & Fri & Sat \\
\midrule
\endhead
FB & 30 & 43 & 55 & 89 & 71 & 52 & 42 \\
Twitter & 60 & 32 & 86 & 44 & 21 & 30 & 28 \\
\bottomrule
\end{longtable}

Continue this problem from last week by creating vectors for fb and
twitter containing the likes/follows for each day. Assign the days of
the week (``Sunday'', ``Monday'', etc.) as names for your vectors.

\begin{Shaded}
\begin{Highlighting}[]
\NormalTok{fb }\OtherTok{\textless{}{-}} \FunctionTok{c}\NormalTok{(}\DecValTok{30}\NormalTok{,}\DecValTok{43}\NormalTok{,}\DecValTok{55}\NormalTok{,}\DecValTok{89}\NormalTok{,}\DecValTok{71}\NormalTok{,}\DecValTok{52}\NormalTok{,}\DecValTok{42}\NormalTok{)}
\NormalTok{twitter }\OtherTok{\textless{}{-}} \FunctionTok{c}\NormalTok{(}\DecValTok{60}\NormalTok{,}\DecValTok{32}\NormalTok{,}\DecValTok{86}\NormalTok{,}\DecValTok{44}\NormalTok{,}\DecValTok{21}\NormalTok{,}\DecValTok{30}\NormalTok{,}\DecValTok{28}\NormalTok{)}

\NormalTok{days }\OtherTok{\textless{}{-}} \FunctionTok{c}\NormalTok{(}\StringTok{"Sun"}\NormalTok{, }\StringTok{"Mon"}\NormalTok{, }\StringTok{"Tues"}\NormalTok{, }\StringTok{"Wed"}\NormalTok{, }\StringTok{"Thurs"}\NormalTok{, }\StringTok{"Fri"}\NormalTok{, }\StringTok{"Sat"}\NormalTok{)}

\FunctionTok{names}\NormalTok{(fb) }\OtherTok{\textless{}{-}}\NormalTok{ days}
\FunctionTok{names}\NormalTok{(twitter) }\OtherTok{\textless{}{-}}\NormalTok{ days}
\end{Highlighting}
\end{Shaded}

\textbf{VS1.} Print just the new fb number of likes/follows from Monday.
What about just the likes/follows from the last day in the data set? Can
you see which day of the week the last day is (supposing you didn't know
it was Saturday)?

\textbf{Your answer:} See the code below.

\begin{Shaded}
\begin{Highlighting}[]
\CommentTok{\#Monday is the second index therefore:}
\NormalTok{fb[}\DecValTok{2}\NormalTok{]}
\end{Highlighting}
\end{Shaded}

\begin{verbatim}
## Mon 
##  43
\end{verbatim}

\begin{Shaded}
\begin{Highlighting}[]
\CommentTok{\#or we can use characters such as:}
\NormalTok{fb[}\StringTok{"Mon"}\NormalTok{]}
\end{Highlighting}
\end{Shaded}

\begin{verbatim}
## Mon 
##  43
\end{verbatim}

\begin{Shaded}
\begin{Highlighting}[]
\CommentTok{\#We can check if they are equal }
\FunctionTok{identical}\NormalTok{(fb[}\DecValTok{2}\NormalTok{],fb[}\StringTok{"Mon"}\NormalTok{])}
\end{Highlighting}
\end{Shaded}

\begin{verbatim}
## [1] TRUE
\end{verbatim}

\begin{Shaded}
\begin{Highlighting}[]
\CommentTok{\#if we don\textquotesingle{}t know the last day is Saturday we can use index since there are}
\CommentTok{\# 7 days in a week.}
\NormalTok{fb[}\DecValTok{7}\NormalTok{] }\CommentTok{\#this will print the Saturday\textquotesingle{}s value}
\end{Highlighting}
\end{Shaded}

\begin{verbatim}
## Sat 
##  42
\end{verbatim}

\begin{Shaded}
\begin{Highlighting}[]
\CommentTok{\#or we can use the tail function:}
\FunctionTok{tail}\NormalTok{(fb, }\DecValTok{1}\NormalTok{)}
\end{Highlighting}
\end{Shaded}

\begin{verbatim}
## Sat 
##  42
\end{verbatim}

\textbf{VS2.} Print just the weekends' number of fb likes/follows. Can
you show two different ways to do this?

\textbf{Your answer:}

\begin{Shaded}
\begin{Highlighting}[]
\CommentTok{\#We can add Saturday + Sunday using their vertex}
\NormalTok{weekend\_fb\_like }\OtherTok{\textless{}{-}}\NormalTok{ fb[}\DecValTok{1}\NormalTok{] }\SpecialCharTok{+}\NormalTok{ fb[}\DecValTok{7}\NormalTok{]}
\NormalTok{weekend\_fb\_like }\CommentTok{\#print the variable to see the value}
\end{Highlighting}
\end{Shaded}

\begin{verbatim}
## Sun 
##  72
\end{verbatim}

\begin{Shaded}
\begin{Highlighting}[]
\CommentTok{\#Using tail and head functions:}
\NormalTok{weekend\_fb\_like1 }\OtherTok{\textless{}{-}} \FunctionTok{tail}\NormalTok{(fb, }\DecValTok{1}\NormalTok{) }\SpecialCharTok{+} \FunctionTok{head}\NormalTok{(fb,}\DecValTok{1}\NormalTok{)}
\NormalTok{weekend\_fb\_like1}
\end{Highlighting}
\end{Shaded}

\begin{verbatim}
## Sat 
##  72
\end{verbatim}

\begin{Shaded}
\begin{Highlighting}[]
\CommentTok{\#We can add Saturday + Sunday using their names}
\NormalTok{weekend\_fb\_like2 }\OtherTok{\textless{}{-}}\NormalTok{ fb[}\StringTok{"Sat"}\NormalTok{] }\SpecialCharTok{+}\NormalTok{ fb[}\StringTok{"Sun"}\NormalTok{]}
\NormalTok{weekend\_fb\_like2}
\end{Highlighting}
\end{Shaded}

\begin{verbatim}
## Sat 
##  72
\end{verbatim}

\textbf{VS3.} Find which days had more than 50 new likes/follows on FB.

\textbf{Your answer:} Tuesday, Wednesday, Thursday, Friday

\begin{Shaded}
\begin{Highlighting}[]
\NormalTok{fb[}\DecValTok{1}\SpecialCharTok{:}\DecValTok{7}\NormalTok{] }\SpecialCharTok{\textgreater{}} \DecValTok{50}
\end{Highlighting}
\end{Shaded}

\begin{verbatim}
##   Sun   Mon  Tues   Wed Thurs   Fri   Sat 
## FALSE FALSE  TRUE  TRUE  TRUE  TRUE FALSE
\end{verbatim}

\begin{Shaded}
\begin{Highlighting}[]
\CommentTok{\#return TRUE if the day has more than 50 new likes/follows on FB}
\end{Highlighting}
\end{Shaded}

\textbf{VS4.} Let's define a day as ``Facebook favorite'' if there were
more than 50 new likes/follows on FB and fewer than 31 new likes/follows
on twitter. Determine whether each day in our data set is a facebook
favorite.

Your output should be a vector of TRUE's and FALSE's, corresponding to
each day of the week.

\textbf{Your answer:} Thursday, Friday

\begin{Shaded}
\begin{Highlighting}[]
\NormalTok{facebook\_favorite }\OtherTok{\textless{}{-}}\NormalTok{ (fb[}\DecValTok{1}\SpecialCharTok{:}\DecValTok{7}\NormalTok{] }\SpecialCharTok{\textgreater{}} \DecValTok{50}\NormalTok{) }\SpecialCharTok{\&}\NormalTok{ (twitter[}\DecValTok{1}\SpecialCharTok{:}\DecValTok{7}\NormalTok{] }\SpecialCharTok{\textless{}} \DecValTok{31}\NormalTok{)}
\NormalTok{facebook\_favorite}
\end{Highlighting}
\end{Shaded}

\begin{verbatim}
##   Sun   Mon  Tues   Wed Thurs   Fri   Sat 
## FALSE FALSE FALSE FALSE  TRUE  TRUE FALSE
\end{verbatim}

\textbf{VS5.} Now print out the number of new fb likes/follows, only for
days which are facebook favorites.

\textbf{Your answer:} 123

\begin{Shaded}
\begin{Highlighting}[]
\CommentTok{\#Thursday + Friday}
\NormalTok{fb[}\DecValTok{5}\NormalTok{] }\SpecialCharTok{+}\NormalTok{ fb[}\DecValTok{6}\NormalTok{]}
\end{Highlighting}
\end{Shaded}

\begin{verbatim}
## Thurs 
##   123
\end{verbatim}

\hypertarget{matrix-exercises}{%
\section{Matrix exercises}\label{matrix-exercises}}

A local animal rescue group is trying to track the effectiveness of
their social media presence; they are currently interested in tracking
follower growth. The table below summarizes the number of page likes or
new followers each day:

\begin{longtable}[]{@{}llllllll@{}}
\toprule
& Sun & Mon & Tues & Wed & Thurs & Fri & Sat \\
\midrule
\endhead
FB & 30 & 43 & 55 & 89 & 71 & 52 & 42 \\
Twitter & 60 & 32 & 86 & 44 & 21 & 30 & 28 \\
\bottomrule
\end{longtable}

\textbf{ME1.} Create a matrix with 7 rows and 2 columns containing the
number of new follows for facebook (column 1) and twitter (column 2).
Name the rows and columns appropriately (``Mon'', ``Tues'', etc. for
each row, and ``Facebook'', ``Twitter'' for the columns).

Save this matrix as ``matrix1'', and then print it.

Hint 1: This is easiest to do by starting over again with the raw data;
it may actually be more challenging to use the named vectors you created
earlier.

Hint 2: Break this problem down into smaller pieces. First create the
matrix (there is starter code at the very end of this document if you
want it; intermediate students should try it on their own first). Then
update your code to save the matrix. Then update your code to add column
names, etc.

\textbf{Your answer:}

Note to myself 1: You can also use dimnames = list(c(``X'',``Y'',``Z''),
c(``A'',``B'',``C''))) to name the columns and rows when creating the
matrix.

Ex: \textgreater{} x \textless- matrix(1:9, nrow = 3, dimnames =
list(c(``X'',``Y'',``Z''), c(``A'',``B'',``C'')))

A B C X 1 4 7 Y 2 5 8 Z 3 6 9

Note to myself 2: The matrix is filled column-wise. This can be reversed
to row-wise filling by passing TRUE to the argument byrow.

Ex: \textgreater{} matrix(1:9, nrow=3, byrow=TRUE) \# fill matrix
row-wise

{[},1{]} {[},2{]} {[},3{]} {[}1,{]} 1 2 3 {[}2,{]} 4 5 6 {[}3,{]} 7 8 9

\begin{Shaded}
\begin{Highlighting}[]
\CommentTok{\#Set number of rows and columns}
\NormalTok{matrix1 }\OtherTok{\textless{}{-}} \FunctionTok{matrix}\NormalTok{(}\FunctionTok{c}\NormalTok{(fb,twitter), }\AttributeTok{nrow =} \DecValTok{7}\NormalTok{, }\AttributeTok{ncol =} \DecValTok{2}\NormalTok{)}

\CommentTok{\#Set the row names}
\FunctionTok{rownames}\NormalTok{(matrix1) }\OtherTok{\textless{}{-}} \FunctionTok{c}\NormalTok{(}\StringTok{"Sun"}\NormalTok{,}\StringTok{"Mon"}\NormalTok{,}\StringTok{"Tues"}\NormalTok{,}\StringTok{"Wed"}\NormalTok{,}\StringTok{"Thurs"}\NormalTok{,}\StringTok{"Fri"}\NormalTok{,}\StringTok{"Sat"}\NormalTok{)}

\CommentTok{\#Set the columns names}
\FunctionTok{colnames}\NormalTok{(matrix1) }\OtherTok{\textless{}{-}} \FunctionTok{c}\NormalTok{(}\StringTok{"Facebook"}\NormalTok{,}\StringTok{"Twitter"}\NormalTok{)}

\CommentTok{\#print the matrix}
\NormalTok{matrix1}
\end{Highlighting}
\end{Shaded}

\begin{verbatim}
##       Facebook Twitter
## Sun         30      60
## Mon         43      32
## Tues        55      86
## Wed         89      44
## Thurs       71      21
## Fri         52      30
## Sat         42      28
\end{verbatim}

\textbf{ME2.} Add a column for instagram, and save the result as
\texttt{matrix2}.

\begin{longtable}[]{@{}llllllll@{}}
\toprule
& Sun & Mon & Tues & Wed & Thurs & Fri & Sat \\
\midrule
\endhead
Insta & 45 & 68 & 25 & 76 & 50 & 41 & 44 \\
\bottomrule
\end{longtable}

Hint: You can either re-write your code to create a new matrix, or
Google how to add a new column to a matrix. There is a function that
will do this.

\textbf{Your answer:} We can use function rbind() to add the row to any
existing matrix. cbind() to add a new column.

\begin{Shaded}
\begin{Highlighting}[]
\NormalTok{matrix2 }\OtherTok{\textless{}{-}} \FunctionTok{cbind}\NormalTok{(matrix1, }\FunctionTok{c}\NormalTok{(}\DecValTok{45}\NormalTok{,}\DecValTok{68}\NormalTok{,}\DecValTok{25}\NormalTok{,}\DecValTok{76}\NormalTok{,}\DecValTok{50}\NormalTok{,}\DecValTok{41}\NormalTok{,}\DecValTok{44}\NormalTok{))}
\FunctionTok{colnames}\NormalTok{(matrix2) }\OtherTok{\textless{}{-}} \FunctionTok{c}\NormalTok{(}\StringTok{"Facebook"}\NormalTok{,}\StringTok{"Twitter"}\NormalTok{, }\StringTok{"Instagram"}\NormalTok{)}
\NormalTok{matrix2}
\end{Highlighting}
\end{Shaded}

\begin{verbatim}
##       Facebook Twitter Instagram
## Sun         30      60        45
## Mon         43      32        68
## Tues        55      86        25
## Wed         89      44        76
## Thurs       71      21        50
## Fri         52      30        41
## Sat         42      28        44
\end{verbatim}

\textbf{ME3.} Use the transpose operator \texttt{t()} so that the rows
represent the social media outlet and the columns represent the days of
the week. (You may want to look up this function by typing \texttt{?t})

Save this as a matrix called \texttt{social}. It should have three rows
(Facebook, Twitter, Instagram) and seven columns (for each day of the
week).

\textbf{Your answer:}

\begin{Shaded}
\begin{Highlighting}[]
\NormalTok{social }\OtherTok{\textless{}{-}} \FunctionTok{t}\NormalTok{(matrix2)}
\NormalTok{social}
\end{Highlighting}
\end{Shaded}

\begin{verbatim}
##           Sun Mon Tues Wed Thurs Fri Sat
## Facebook   30  43   55  89    71  52  42
## Twitter    60  32   86  44    21  30  28
## Instagram  45  68   25  76    50  41  44
\end{verbatim}

\textbf{ME4.} Suppose we want to double the number of follows each day.
Multiply your matrix \texttt{social} by 2 with regular multiplication
(*). Does it work?

\textbf{Your answer:} Yes, it works.

\begin{Shaded}
\begin{Highlighting}[]
\NormalTok{social }\SpecialCharTok{*} \DecValTok{2}
\end{Highlighting}
\end{Shaded}

\begin{verbatim}
##           Sun Mon Tues Wed Thurs Fri Sat
## Facebook   60  86  110 178   142 104  84
## Twitter   120  64  172  88    42  60  56
## Instagram  90 136   50 152   100  82  88
\end{verbatim}

\textbf{ME5.} Using your \texttt{social} matrix, get the Facebook
follows from Wednesday. We can subset a matrix using the syntax:
\texttt{my\_matrix{[}row,\ column{]}}.

\textbf{Your answer:} 89

\begin{Shaded}
\begin{Highlighting}[]
\NormalTok{fb\_wed }\OtherTok{\textless{}{-}}\NormalTok{ social[}\StringTok{"Facebook"}\NormalTok{, }\StringTok{"Wed"}\NormalTok{]}
\NormalTok{fb\_wed}
\end{Highlighting}
\end{Shaded}

\begin{verbatim}
## [1] 89
\end{verbatim}

\textbf{ME6.} Use R code to print just the Monday reactions from your
\texttt{social} matrix. This should be a column with 3 entries.

\textbf{Your answer:}

\begin{Shaded}
\begin{Highlighting}[]
\NormalTok{monday\_likes }\OtherTok{\textless{}{-}}\NormalTok{ social[, }\StringTok{"Mon"}\NormalTok{]}
\NormalTok{monday\_likes}
\end{Highlighting}
\end{Shaded}

\begin{verbatim}
##  Facebook   Twitter Instagram 
##        43        32        68
\end{verbatim}

\textbf{ME7 OPTIONAL Intermediate Challenge.} Remove the row with
facebook follows from the matrix; the remaining matrix should only
contain rows for twitter and instagram.

\textbf{Your answer:}

\textbf{ME8 OPTIONAL Intermediate Challenge.} (requires some knowledge
of matrix multiplication): A marketer gets paid 5 cents per new follow
on weekday and 8 cents per follow on weekends. Find the total weekly
amount the marketer gets paid per social media outlet.

\textbf{Your answer:}

\textbf{ME9 OPTIONAL Intermediate Challenge.} How can you swap the order
of the rows?

\textbf{Your answer:}

\hypertarget{data-frames-exercises}{%
\section{Data frames exercises}\label{data-frames-exercises}}

R has many built in data sets! Let's look at one of them.

\textbf{DFE1.} Look up the \texttt{iris} data set in R using the help
function (or Google). What is this data? How many observations does it
have?

\textbf{Your answer:} ris data set gives the measurements in centimeters
of the variables sepal length and width and petal length and width,
respectively, for 50 flowers from each of 3 species of iris. The species
are Iris setosa, versicolor, and virginica. The dataset contains 150
observations with 5 attributes named as Sepal width, Sepal length, Petal
width, Petal length and flower type.

\begin{Shaded}
\begin{Highlighting}[]
\NormalTok{?iris}
\end{Highlighting}
\end{Shaded}

\begin{verbatim}
## starting httpd help server ... done
\end{verbatim}

\textbf{DFE2.} Compare the results of calling \texttt{typeof()},
\texttt{class()}, and \texttt{str()} on the iris data set. Which one(s)
are most useful?

\textbf{Your answer:} typeof() determines the (R internal) type or
storage mode of any object. It is useful to see the data type of the
data set. class() function is useful if we want to see the class of the
data set. str() gives an overview of the data set. It is useful to see
the structure of the data set.

\begin{Shaded}
\begin{Highlighting}[]
\FunctionTok{typeof}\NormalTok{(iris)}
\end{Highlighting}
\end{Shaded}

\begin{verbatim}
## [1] "list"
\end{verbatim}

\begin{Shaded}
\begin{Highlighting}[]
\FunctionTok{class}\NormalTok{(iris)}
\end{Highlighting}
\end{Shaded}

\begin{verbatim}
## [1] "data.frame"
\end{verbatim}

\begin{Shaded}
\begin{Highlighting}[]
\FunctionTok{str}\NormalTok{(iris)}
\end{Highlighting}
\end{Shaded}

\begin{verbatim}
## 'data.frame':    150 obs. of  5 variables:
##  $ Sepal.Length: num  5.1 4.9 4.7 4.6 5 5.4 4.6 5 4.4 4.9 ...
##  $ Sepal.Width : num  3.5 3 3.2 3.1 3.6 3.9 3.4 3.4 2.9 3.1 ...
##  $ Petal.Length: num  1.4 1.4 1.3 1.5 1.4 1.7 1.4 1.5 1.4 1.5 ...
##  $ Petal.Width : num  0.2 0.2 0.2 0.2 0.2 0.4 0.3 0.2 0.2 0.1 ...
##  $ Species     : Factor w/ 3 levels "setosa","versicolor",..: 1 1 1 1 1 1 1 1 1 1 ...
\end{verbatim}

\textbf{DFE3.} Print the first 3 rows of the iris data set.

\textbf{Your answer:}

\begin{Shaded}
\begin{Highlighting}[]
\FunctionTok{head}\NormalTok{(iris,}\DecValTok{3}\NormalTok{)}
\end{Highlighting}
\end{Shaded}

\begin{verbatim}
##   Sepal.Length Sepal.Width Petal.Length Petal.Width Species
## 1          5.1         3.5          1.4         0.2  setosa
## 2          4.9         3.0          1.4         0.2  setosa
## 3          4.7         3.2          1.3         0.2  setosa
\end{verbatim}

\textbf{DFE4.} You can access just one column from a data frame using
the \texttt{\$} operators:

\texttt{my\_data\$column\_name}

Use this to print just the ``Species'' column from the iris data set.

\textbf{Your answer:}

\begin{Shaded}
\begin{Highlighting}[]
\NormalTok{iris}\SpecialCharTok{$}\NormalTok{Species}
\end{Highlighting}
\end{Shaded}

\begin{verbatim}
##   [1] setosa     setosa     setosa     setosa     setosa     setosa    
##   [7] setosa     setosa     setosa     setosa     setosa     setosa    
##  [13] setosa     setosa     setosa     setosa     setosa     setosa    
##  [19] setosa     setosa     setosa     setosa     setosa     setosa    
##  [25] setosa     setosa     setosa     setosa     setosa     setosa    
##  [31] setosa     setosa     setosa     setosa     setosa     setosa    
##  [37] setosa     setosa     setosa     setosa     setosa     setosa    
##  [43] setosa     setosa     setosa     setosa     setosa     setosa    
##  [49] setosa     setosa     versicolor versicolor versicolor versicolor
##  [55] versicolor versicolor versicolor versicolor versicolor versicolor
##  [61] versicolor versicolor versicolor versicolor versicolor versicolor
##  [67] versicolor versicolor versicolor versicolor versicolor versicolor
##  [73] versicolor versicolor versicolor versicolor versicolor versicolor
##  [79] versicolor versicolor versicolor versicolor versicolor versicolor
##  [85] versicolor versicolor versicolor versicolor versicolor versicolor
##  [91] versicolor versicolor versicolor versicolor versicolor versicolor
##  [97] versicolor versicolor versicolor versicolor virginica  virginica 
## [103] virginica  virginica  virginica  virginica  virginica  virginica 
## [109] virginica  virginica  virginica  virginica  virginica  virginica 
## [115] virginica  virginica  virginica  virginica  virginica  virginica 
## [121] virginica  virginica  virginica  virginica  virginica  virginica 
## [127] virginica  virginica  virginica  virginica  virginica  virginica 
## [133] virginica  virginica  virginica  virginica  virginica  virginica 
## [139] virginica  virginica  virginica  virginica  virginica  virginica 
## [145] virginica  virginica  virginica  virginica  virginica  virginica 
## Levels: setosa versicolor virginica
\end{verbatim}

\textbf{DFE5.} Write a logical expression to test whether the first
plant's species in the Species column is ``setosa''. Then try to write a
logical expression to test whether the first plant's species is setosa
AND its petal length is longer than 1 cm.

Hint: You need to use the skills/code from the previous question.

\textbf{Your answer:} Yes, the first plant is setosa and its petal
length is longer than 1 cm.

\begin{Shaded}
\begin{Highlighting}[]
\CommentTok{\#First part of the question}
\NormalTok{iris}\SpecialCharTok{$}\NormalTok{Species[}\DecValTok{1}\NormalTok{] }\SpecialCharTok{==} \StringTok{"setosa"}
\end{Highlighting}
\end{Shaded}

\begin{verbatim}
## [1] TRUE
\end{verbatim}

\begin{Shaded}
\begin{Highlighting}[]
\CommentTok{\#Second part of the question}
\NormalTok{iris}\SpecialCharTok{$}\NormalTok{Species[}\DecValTok{1}\NormalTok{] }\SpecialCharTok{==} \StringTok{"setosa"} \SpecialCharTok{\&\&}\NormalTok{ iris}\SpecialCharTok{$}\NormalTok{Petal.Length }\SpecialCharTok{\textgreater{}} \DecValTok{1}
\end{Highlighting}
\end{Shaded}

\begin{verbatim}
## [1] TRUE
\end{verbatim}

\textbf{DFE6. OPTIONAL Intermediate} Print only the rows from the iris
data set for plants whose Species is ``versicolor''. Then write code to
determine how many plants fit this condition.

\textbf{Your answer:}

\hypertarget{optional-intermediate-list-exercises-not-graded-just-for-fun-if-you-completed-the-intermediate-reading}{%
\section{OPTIONAL Intermediate List exercises (not graded; just for fun
if you completed the intermediate
reading)}\label{optional-intermediate-list-exercises-not-graded-just-for-fun-if-you-completed-the-intermediate-reading}}

\textbf{LE1 OPTIONAL Intermediate.} Suppose we have data on a book,
including the title (``Game of Thrones''), year of publication (1996),
and whether the series is complete (FALSE). What happens if we try to
store these three pieces of information in a vector?

\textbf{Your answer:}

\textbf{LE2 OPTIONAL Intermediate.} Create a list to store the three
pieces of information given. Use \texttt{list()} to create a new list
and call it \texttt{book}.

\textbf{Your answer:}

\textbf{LE3 OPTIONAL Intermediate.} Edit your list \texttt{book} from
the previous exercise so that the first element is called title, the
second element is named year, and the third element is named
is\_complete.

\textbf{Your answer:}

\textbf{LE4 OPTIONAL Intermediate.} Use the book list, select just the
title. How many ways can you think of to do this?

\textbf{Your answer:}

\textbf{LE5 OPTIONAL Intermediate.} Now select just the \textbf{first}
available format (not the entire vector of all possible formats).

\textbf{Your answer:}

\hypertarget{hint-for-matrix-exercise-1-me1}{%
\paragraph{Hint for Matrix Exercise 1
(ME1)}\label{hint-for-matrix-exercise-1-me1}}

Starter code to create the matrix:

\begin{Shaded}
\begin{Highlighting}[]
\NormalTok{fb }\OtherTok{\textless{}{-}} \FunctionTok{c}\NormalTok{(}\DecValTok{30}\NormalTok{,}\DecValTok{43}\NormalTok{,}\DecValTok{55}\NormalTok{,}\DecValTok{89}\NormalTok{,}\DecValTok{71}\NormalTok{,}\DecValTok{52}\NormalTok{,}\DecValTok{42}\NormalTok{)}
\NormalTok{twitter }\OtherTok{\textless{}{-}} \FunctionTok{c}\NormalTok{(}\DecValTok{60}\NormalTok{,}\DecValTok{32}\NormalTok{,}\DecValTok{86}\NormalTok{,}\DecValTok{44}\NormalTok{,}\DecValTok{21}\NormalTok{,}\DecValTok{30}\NormalTok{,}\DecValTok{28}\NormalTok{)}

\CommentTok{\# modify to set the correct number of rows and columns}
\NormalTok{matrix1 }\OtherTok{\textless{}{-}} \FunctionTok{matrix}\NormalTok{(}\FunctionTok{c}\NormalTok{(fb, twitter), }\AttributeTok{byrow =} \ConstantTok{FALSE}\NormalTok{)}

\CommentTok{\# add code to set the row names}

\CommentTok{\# add code to set the column names}
\end{Highlighting}
\end{Shaded}


\end{document}
